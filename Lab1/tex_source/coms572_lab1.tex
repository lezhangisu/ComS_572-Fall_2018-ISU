\documentclass{article}

\usepackage{fancyhdr}
\usepackage{extramarks}
\usepackage{amsmath}
\usepackage{amsthm}
\usepackage{amsfonts}
\usepackage{tikz}
\usepackage[plain]{algorithm}
\usepackage{algpseudocode}
\usepackage{enumitem}
\usepackage{listings}

\usetikzlibrary{automata,positioning}

%
% Basic Document Settings
%

\lstset{
mathescape=true
% basicstyle=\small\ttfamily,
% columns=flexible,
% breaklines=true
}

\topmargin=-0.45in
\evensidemargin=0in
\oddsidemargin=0in
\textwidth=6.5in
\textheight=9.0in
\headsep=0.25in

\linespread{1.1}

\pagestyle{fancy}
\lhead{\hmwkAuthorName}
\chead{\hmwkClass\ : \hmwkTitle}
\rhead{\firstxmark}
\lfoot{\lastxmark}
\cfoot{\thepage}

\renewcommand\headrulewidth{0.4pt}
\renewcommand\footrulewidth{0.4pt}

\setlength\parindent{0pt}

%
% Create Problem Sections
%

\newcommand{\enterProblemHeader}[1]{
    \nobreak\extramarks{}{Problem \arabic{#1} continued on next page\ldots}\nobreak{}
    \nobreak\extramarks{Problem \arabic{#1} (continued)}{Problem \arabic{#1} continued on next page\ldots}\nobreak{}
}

\newcommand{\exitProblemHeader}[1]{
    \nobreak\extramarks{Problem \arabic{#1} (continued)}{Problem \arabic{#1} continued on next page\ldots}\nobreak{}
    \stepcounter{#1}
    \nobreak\extramarks{Problem \arabic{#1}}{}\nobreak{}
}

\setcounter{secnumdepth}{0}
\newcounter{partCounter}
\newcounter{homeworkProblemCounter}
\setcounter{homeworkProblemCounter}{1}
\nobreak\extramarks{Problem \arabic{homeworkProblemCounter}}{}\nobreak{}

%
% Homework Problem Environment
%
% This environment takes an optional argument. When given, it will adjust the
% problem counter. This is useful for when the problems given for your
% assignment aren't sequential. See the last 3 problems of this template for an
% example.
%
\newenvironment{homeworkProblem}[1][-1]{
    \ifnum#1>0
        \setcounter{homeworkProblemCounter}{#1}
    \fi
    \section{Problem \arabic{homeworkProblemCounter}}
    \setcounter{partCounter}{1}
    \enterProblemHeader{homeworkProblemCounter}
}{
    \exitProblemHeader{homeworkProblemCounter}
}

%
% Homework Details
%   - Title
%   - Due date
%   - Class
%   - Section/Time
%   - Instructor
%   - Author
%

\newcommand{\hmwkTitle}{Lab\ \#1}
\newcommand{\hmwkDueDate}{September 28, 2018}
\newcommand{\hmwkClass}{COMS 572}
\newcommand{\hmwkClassTime}{}
\newcommand{\hmwkClassInstructor}{Professor Jin Tian}
\newcommand{\hmwkAuthorName}{Le Zhang}

%
% Title Page
%

\title{
    \vspace{2in}
    \textmd{\textbf{\hmwkClass:\ \hmwkTitle}}\\
    \normalsize\vspace{0.1in}\small{\hmwkDueDate\ by 17:00pm}\\
    \vspace{0.1in}\large{\textit{\hmwkClassInstructor\ \hmwkClassTime}}
    \vspace{3in}
}

\author{\textbf{\hmwkAuthorName}}
\date{}

\renewcommand{\part}[1]{\textbf{\large Part \Alph{partCounter}}\stepcounter{partCounter}\\}

%
% Various Helper Commands
%

% Useful for algorithms
\newcommand{\alg}[1]{\textsc{\bfseries \footnotesize #1}}

% For derivatives
\newcommand{\deriv}[1]{\frac{\mathrm{d}}{\mathrm{d}x} (#1)}

% For partial derivatives
\newcommand{\pderiv}[2]{\frac{\partial}{\partial #1} (#2)}

% Integral dx
\newcommand{\dx}{\mathrm{d}x}

% Alias for the Solution section header
\newcommand{\solution}{\textbf{\large Solution}}

% Cartesian product
\newcommand{\Cross}{\mathbin{\tikz [x=1.4ex,y=1.4ex,line width=.2ex] \draw (0,0) -- (1,1) (0,1) -- (1,0);}}%

% Probability commands: Expectation, Variance, Covariance, Bias
\newcommand{\E}{\mathrm{E}}
\newcommand{\Var}{\mathrm{Var}}
\newcommand{\Cov}{\mathrm{Cov}}
\newcommand{\Bias}{\mathrm{Bias}}

\begin{document}

\maketitle

%
% Problem 1
%
\pagebreak
\begin{homeworkProblem}
The sample intranets 1, 5, and 7 have been randomly created, with the following propensities (from weakest to strongest):

\begin{itemize}
\item the more QUERY words on a page, the more likely the links on that page lead to the goal node
\item the more QUERY words in the hypertext associated with a hyperlink, the more likely that hyperlink leads to the goal node
\item the more consecutive and in numerical order QUERY words there are in a hyperlink, the more likely that hyperlink leads to the goal node (eg, seeing QUERY1 QUERY2 QUERY3 is a very good indicator)
\end{itemize}

\textbf{QUESTIONS:}

\begin{enumerate}[label=\alph*.]
    \item[2a.] Use the above information to devise a heuristic function for use in best-first search. Describe your motivation for your heuristic. Note that unlike the standard approach where the heuristic is applied to the next state, here we want to use our heuristic to decide which hyperlink to 'click on' (fetch) next. This means that your heuristic function scores each arc (hyperlink) coming out of the current node and not each child node. Is your heuristic admissible? Explain why or why not. (You're not required to write an admissible heuristic.)
    
    \textbf{Answer:}
    
    As described in the question, we have 3 levels of propensities from weakest to strongest. Thus, we want to assign scores to those links based on those propensities and then rank them by those scores. Those propensities have different importance so we need to put different weight on them. In the mean while, we don't want to mix scores from different levels. By doing some experiments, I found that the number of QUERY words on a single page is no more than 2 digits and number of QUERY words in the hypertext should be 1 digit only. Here I use $x_1, x_2,$ and $x_3$ to reperesent the number of QUERY words on a page, number of QUERY words in hypertext, and length of consecutive QUERY words in numerical order, respectively. Therefore, for a node $A$ my heuristic function is like:
    
    $$
    h(A) = x_1 + 100x_2 + 1000x_3
    $$
    
    In my opinion, my heuristic function is admissible. Because the strongest propensity takes control anytime before the weaker ones. And different levels of scores never mix with each other so links with higher probability of finding the goal will always be explored first. As a result, the heuristic function never overestimate the cost to reach the goal in this problem. Therefore, I'd say it is admissible. 
    
    \item[2c.] How well did your heuristic work on the sample intranets
    
    \textbf{Answer:}
    
    Here is the summary of results I got from my program:
    
\begin{tabular}{lllll|l|lll}
intranet\# & BFS  & DFS   & BEST & BEAM & $\Leftarrow$ My results        & BFS  & DFS   & BEST \\
1          & 91/4 & 58/15 & 18/7 & 18/7 &                                & 91/4 & 58/15 & 19/7 \\
5          & 88/8 & 42/10 & 25/9 & 25/9 &                                & 88/8 & 42/10 & 29/9 \\
7          & 56/6 & 12/9  & 27/6 & 27/6 &  Sample Results $\Rightarrow$  & 56/6 & 12/9  & 27/8
\end{tabular}

    As we can see from the results, my heuristic function (BEST search algorithm) outperforms other algorithms. Compared with sample results given in the dataset, my heuristic algorithm outperforms the sample results as well (Intranet\#1: [18/7 vs. 19/7]; Intranet\#5: [25/9 vs. 29/9]; Intranet\#7: [27/6 vs. 27/8]). The reason BEAM algorithm does not improve from BEST algorithm is that the intranets given are a little small and the beam width (20) is kind of large for this dataset. If we have a larger dataset or a smaller beam width, BEAM will outperform BEST to some degree. 
    
    \pagebreak
    \textbf{Solution Paths Found:}
\begin{verbatim}
***** Intranet_1 *****
========BFS========
result  ['page1.html', 'page18.html', 'page29.html', 'page99.html', 'page50.html']
========DFS========
result  ['page1.html', 'page23.html', 'page60.html', 'page39.html', 'page78.html', 
'page25.html', 'page42.html', 'page84.html', 'page30.html', 'page68.html', 'page93.html', 
'page87.html', 'page79.html', 'page2.html', 'page83.html', 'page50.html']
========BEST========
result  ['page1.html', 'page14.html', 'page69.html', 'page87.html', 'page79.html', 
'page2.html', 'page83.html', 'page50.html']
========BEAM========
result  ['page1.html', 'page14.html', 'page69.html', 'page87.html', 'page79.html', 
'page2.html', 'page83.html', 'page50.html']

***** Intranet_5 *****
========BFS========
result  ['page1.html', 'page40.html', 'page99.html', 'page89.html', 'page87.html', 
'page96.html', 'page95.html', 'page72.html', 'page62.html']
========DFS========
result  ['page1.html', 'page40.html', 'page99.html', 'page5.html', 'page97.html', 
'page68.html', 'page48.html', 'page7.html', 'page95.html', 'page72.html', 'page62.html']
========BEST========
result  ['page1.html', 'page40.html', 'page99.html', 'page88.html', 'page19.html', 
'page42.html', 'page35.html', 'page95.html', 'page72.html', 'page62.html']
========BEAM========
result  ['page1.html', 'page40.html', 'page99.html', 'page88.html', 'page19.html', 
'page42.html', 'page35.html', 'page95.html', 'page72.html', 'page62.html']

***** Intranet_7 *****
========BFS========
result  ['page1.html', 'page48.html', 'page71.html', 'page57.html', 'page62.html', 
'page61.html', 'page86.html']
========DFS========
result  ['page1.html', 'page48.html', 'page71.html', 'page57.html', 'page90.html', 
'page39.html', 'page60.html', 'page11.html', 'page78.html', 'page86.html']
========BEST========
result  ['page1.html', 'page48.html', 'page71.html', 'page57.html', 'page62.html', 
'page61.html', 'page86.html']
========BEAM========
result  ['page1.html', 'page48.html', 'page71.html', 'page57.html', 'page62.html', 
'page61.html', 'page86.html']
\end{verbatim}

\end{enumerate}

\pagebreak
\end{homeworkProblem}

% 
% Problem 2
% 
\begin{homeworkProblem}
Nodes visited and path found with description of query words and the domain if you did WWW adventure.
    \textbf{Answer:}
    \begin{itemize}
    \item Query words: `Paolo Cesare Maldini'
    \item Website domain: `https://en.wikipedia.org'
    \item Starting page: `/wiki/Sports'
    \item Searching algorithm: Beam Search with beam\_width = 20
    \item Heuristic function: if any of these words appears in the hypertext of a link, take a sum of the corresponding numbers as the score of that link: \{'soccer': 3, 'football': 3, 'paolo': 5, 'maldini': 5, 'serie a': 5, 'italy': 4, 'milan': 5, 'world cup':5, 'association': 3\}
    \item Nodes visited: 9
    \item Path length: 5
    \item Path found: 
    \begin{verbatim}
    ['/wiki/Sports', 
    '/wiki/Association_football', 
    '/wiki/Category:Laws_of_association_football', 
    '/wiki/Penalty_shoot-out_(association_football)', 
    '/wiki/Italy_national_football_team', 
    '/wiki/Paolo_Maldini']
    \end{verbatim}
    \end{itemize} 
    
    Discussions: 
    
    \quad\quad I tried to use BFS and DFS for the WWW searching in the first place but they didn't work well. It was because there are thousands of links on each of the Wikipedia webpages. As a result, it may take hours even days before BFS and DFS can find the target page. 
    
    \quad\quad Best-first search seems to be a good way to do it, however, thousands of links eats the memory rapidly. Therefore, eventually, I picked beam search to finish this task. 
    
    \quad\quad The fact was, when I was using BFS, it visited more than 20,000 pages and the goal page is still not reached. On the other hand, with beam search, only 9 pages were visited before we get to the goal. It is effective and efficient as long as you have the correct heuristic function. 
    
    \quad\quad Just for fun, I also searched for query words `Cristiano Ronaldo dos Santos Aveiro', started from page `/wiki/Main\_page', with similar heuristic function but different values to calculate scores: 
    \begin{verbatim}
    {'soccer': 3, 'football': 3, 'cristiano': 5, 'madrid': 5, 'la liga': 5,
                     'portugal': 4, 'ronaldo': 5, 'world cup':5, 'association': 3}
    \end{verbatim}
    \quad\quad At this time, because it started from a more neutral page ``main page'', it took longer to get to the goal. It visited 35 pages and the length of solution path was 25. It is still a good result which means the heuristic function is very effective and beam search works perfectly with real world cases. 
    
    \quad\quad In summary, the beam search that I used worked fine with real world web searching. There is enough evidence to prove that the heuristic function I applied to my algorithm is effective in practical use. 

    
    
\end{homeworkProblem}


\end{document}
