\documentclass{article}

\usepackage{fancyhdr}
\usepackage{extramarks}
\usepackage{amsmath}
\usepackage{amsthm}
\usepackage{amsfonts}
\usepackage{tikz}
\usepackage[plain]{algorithm}
\usepackage{algpseudocode}
\usepackage{enumitem}
\usepackage{listings}

\usetikzlibrary{automata,positioning}

%
% Basic Document Settings
%

\lstset{%
mathescape=true}

\topmargin=-0.45in
\evensidemargin=0in
\oddsidemargin=0in
\textwidth=6.5in
\textheight=9.0in
\headsep=0.25in

\linespread{1.1}

\pagestyle{fancy}
\lhead{\hmwkAuthorName}
\chead{\hmwkClass\ : \hmwkTitle}
\rhead{\firstxmark}
\lfoot{\lastxmark}
\cfoot{\thepage}

\renewcommand\headrulewidth{0.4pt}
\renewcommand\footrulewidth{0.4pt}

\setlength\parindent{0pt}

%
% Create Problem Sections
%

\newcommand{\enterProblemHeader}[1]{
    \nobreak\extramarks{}{Problem \arabic{#1} continued on next page\ldots}\nobreak{}
    \nobreak\extramarks{Problem \arabic{#1} (continued)}{Problem \arabic{#1} continued on next page\ldots}\nobreak{}
}

\newcommand{\exitProblemHeader}[1]{
    \nobreak\extramarks{Problem \arabic{#1} (continued)}{Problem \arabic{#1} continued on next page\ldots}\nobreak{}
    \stepcounter{#1}
    \nobreak\extramarks{Problem \arabic{#1}}{}\nobreak{}
}

\setcounter{secnumdepth}{0}
\newcounter{partCounter}
\newcounter{homeworkProblemCounter}
\setcounter{homeworkProblemCounter}{1}
\nobreak\extramarks{Problem \arabic{homeworkProblemCounter}}{}\nobreak{}

%
% Homework Problem Environment
%
% This environment takes an optional argument. When given, it will adjust the
% problem counter. This is useful for when the problems given for your
% assignment aren't sequential. See the last 3 problems of this template for an
% example.
%
\newenvironment{homeworkProblem}[1][-1]{
    \ifnum#1>0
        \setcounter{homeworkProblemCounter}{#1}
    \fi
    \section{Problem \arabic{homeworkProblemCounter}}
    \setcounter{partCounter}{1}
    \enterProblemHeader{homeworkProblemCounter}
}{
    \exitProblemHeader{homeworkProblemCounter}
}

%
% Homework Details
%   - Title
%   - Due date
%   - Class
%   - Section/Time
%   - Instructor
%   - Author
%

\newcommand{\hmwkTitle}{Homework\ \#2}
\newcommand{\hmwkDueDate}{September 7, 2018}
\newcommand{\hmwkClass}{COMS 572}
\newcommand{\hmwkClassTime}{}
\newcommand{\hmwkClassInstructor}{Professor Jin Tian}
\newcommand{\hmwkAuthorName}{Le Zhang}

%
% Title Page
%

\title{
    \vspace{2in}
    \textmd{\textbf{\hmwkClass:\ \hmwkTitle}}\\
    \normalsize\vspace{0.1in}\small{\hmwkDueDate\ by 17:00pm}\\
    \vspace{0.1in}\large{\textit{\hmwkClassInstructor\ \hmwkClassTime}}
    \vspace{3in}
}

\author{\textbf{\hmwkAuthorName}}
\date{}

\renewcommand{\part}[1]{\textbf{\large Part \Alph{partCounter}}\stepcounter{partCounter}\\}

%
% Various Helper Commands
%

% Useful for algorithms
\newcommand{\alg}[1]{\textsc{\bfseries \footnotesize #1}}

% For derivatives
\newcommand{\deriv}[1]{\frac{\mathrm{d}}{\mathrm{d}x} (#1)}

% For partial derivatives
\newcommand{\pderiv}[2]{\frac{\partial}{\partial #1} (#2)}

% Integral dx
\newcommand{\dx}{\mathrm{d}x}

% Alias for the Solution section header
\newcommand{\solution}{\textbf{\large Solution}}

% Cartesian product
\newcommand{\Cross}{\mathbin{\tikz [x=1.4ex,y=1.4ex,line width=.2ex] \draw (0,0) -- (1,1) (0,1) -- (1,0);}}%

% Probability commands: Expectation, Variance, Covariance, Bias
\newcommand{\E}{\mathrm{E}}
\newcommand{\Var}{\mathrm{Var}}
\newcommand{\Cov}{\mathrm{Cov}}
\newcommand{\Bias}{\mathrm{Bias}}

\begin{document}

\maketitle

%
% Problem 1
%
\pagebreak
\begin{homeworkProblem}
\textit{(20 pts.)} Give a complete problem formulation for each of the following. Choose a formulation that is precise enough to be implemented.

\begin{enumerate}[label=\alph*.]
    \item[a.] Using only four colors, you have to color a planar map in such a way that no two adjacent regions have the same color.
    
    \textbf{Answer:}
    
    States: Planar map with regions with or wihout colors.
    
    Initial State: Planar map with no regions colored.
    
    Actions: Choose an uncolored region and color it with one of the four colors and make it different from all adjacent regions.
    
    Goal Test: All regions of the map are colored and no two adjacent regions have the same color.
    
    Path Cost: 1 per action.
    
    \item[d.] You have three jugs, measuring 12 gallons, 8 gallons, and 3 gallons, and a water faucet. You can fill the jugs up or empty them out from one to another or onto the ground. You need to measure out exactly one gallon.
    
    \textbf{Answer:}
    
    States: 3 jugs with water, say [i, j, k] (in gallons)
    
    Initial state: No water in all jugs, [0, 0, 0]
    
    Actions:
    
    1. Fill one of the jugs, [12, j, k] or [i, 8, k] or [i, j, 3]; 
    
    2. Empty one of them, [0, j, k] or [i, 0, k] or [i, j, 0]; 
    
    3. For any two jugs A and B, with current water X gallons in A and Y gallons in B, pour water from jug B to jug A; this makes jug A to have water min(X+Y, capacity of jug A) gallons, and jug B to have water (X + Y - min(X+Y, capacity of jug A)). 
    
    Goal test: The amount of water in 3 jugs [i, j, k], where at least one of i, j, k is 1.

    Path Cost: 1 per action.

\end{enumerate}

\end{homeworkProblem}

% \pagebreak

% \begin{thebibliography}{9}
% \bibitem{AI_textbook} 
% Russell, Stuart and Norvig, Peter. 
% \textit{Artificial Intelligence: A Modern Approach}. 
% Prentice Hall Press, Upper Saddle River, NJ, USA, 2009.
 
% \bibitem{wiki_Loebner} 
% \texttt{https://en.wikipedia.org/wiki/Loebner\_Prize}

% \bibitem{wiki_Turing_test} 
% \texttt{https://en.wikipedia.org/wiki/Turing\_test}

% \bibitem{wiki_Mitsuku} 
% \texttt{https://en.wikipedia.org/wiki/Mitsuku}
 
% \bibitem{loabner_prize} 
% \texttt{https://www.aisb.org.uk/events/loebner-prize}

% \bibitem{LP2017}
% \texttt{https://chatbotsmagazine.com/how-to-win-a-turing-test-the-loebner-prize-3ac2752250f1}

% \bibitem{alphagozero}
% Silver, David, et al.
% \textit{Mastering the game of Go without human knowledge}. 
% Nature, Macmillan Publishers Limited, 2017.

% \bibitem{mitsuku}
% \texttt{https://www.pandorabots.com/mitsuku/}

% \end{thebibliography}

\end{document}
